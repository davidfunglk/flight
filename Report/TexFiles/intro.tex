% !TEX root = main.tex

Atlantic hurricanes are tropical cyclones that form in the Atlantic Ocean between the months of June to November every year. These natural weather formations come in varying intensities and can lead to billion of dollars worth of damage to buildings and properties. The main goals of our project involve examining the effect of hurricanes falling into (3,4,5) categories of the Saffir-Simpson Hurricane Wind Scale on the US airport network through changes in flight cancellations and delays corresponding to changes in the weather and geographical distance away from the center of the hurricane. The hurricanes whose behaviror and route we studied in depth were Sandy (10/2012, Class: 3), Ike (09/2008, Class 4), and Katrina (08/2005, Class: 5). 

The stakeholders in our project include airport officials and airlines. For example, an airport might want to foresee the effect of an impending hurricane and how many flights may be canceled. An airline may also want to predict the probability of cancelling its flights based on historical data. This way, they can prepare and react to the situation, sending planes to the right locations to have the least impact on overall service. Consumers indirectly benefit from accurate prediction of flight cancellations since they can make changes to their travel plans earlier, diminishing the impact of such cancellation.